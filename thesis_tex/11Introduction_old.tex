\chapter{Introduction}

\section{Epidemiological Surveillance at the Robert Koch Institute}

\subsection{Epidemiology}
According to the World Health Organization (\GLS{WHO}), epidemiology is a large field of studies that focuses on the distribution and the determinants of health-related signals \citep{WHOepi}.
Among many foci of epidemiology like health and demography, mental health, and non-communicable diseases, the primary interest during to proceedings of this thesis lies on \textbf{infectious epidemiology}.
Its overall goal is the detection and (subsequent) containment of infectious disease outbreaks.
It is important to mention that such disease outbreaks can spread via a direct transmission (contagious disease), a vector like insects (communicable disease), or both ways.
However, not all occurrences of infectious diseases are of interest in the field of infectious epidemiology like a common cold that is typically taken care of by the public health system.
To, therefore, be more specific, epidemiology focuses on diseases occurring as part of epidemics, are likely to turn into an epidemic, or significantly burden the public health system.
A simple definition for epidemics is \textit{``the occurrences of cases more than usual given time and region"}.
For the investigation and prevention of outbreaks, epidemiology is not limiting itself to specific methods but include several families of methods like surveillance and descriptive studies.


\subsubsection{Epidemiological Surveillance}
Infectious epidemiology is firmly linked to an early detection objective to minimize the potential risk to human health \citep{EarlyDetection}.
Therefore, surveillance is an indispensable tool for a functional early warning mechanism.
Through surveillance, epidemiologists can detect all hazard (nuclear, chemical, infectious diseases \ldots) origins menacing public health.
The detection of such threats is then answered by a rapid investigation and containment of further damage.
But, the monitoring of an outbreak is also essential to allocate resources of the public health systems effectively \citep{EarlyDetection}.

Hints for an outbreak can be an increased amount of reported cases of a dangerous infection or changed circumstances that are known to entail disease outbreaks (e.g., a loss of proper sanitation can lead to a cholera outbreak).
Besides the traditional disease-based surveillance that processes laboratory confirmations other systems such as, syndromic surveillance, external factors like weather, attendance monitoring at school and workplace, social media, and the web are also informative \citep{EarlyDetection}.

\paragraph{Event and Indicator-Based Surveillance}

\subparagraph{Epidemic Intelligence}
Independent of the surveillance source, epidemiologists need to systematically detect, verify, and share data to acquire an informative picture of the epidemic threat to global health \citep{EarlyDetection}.
This process is called \textbf{epidemic intelligence}\index{epidemic intelligence} and facilitates epidemiological surveillance through information processing and supply of formal, informal, actively and passively acquired information.


\subparagraph{Event-Based Surveillance}
The monitoring of information not generated by the public health system and its analysis is called \textbf{event-based-surveillance}\index{event-based-surveillance}(\gls{EBS}) and is the speed determining factor in epidemiological surveillance.
Only with EBS, epidemiologists can detect and report events before the recognition of human cases in the reporting system of the public health system \citep{EarlyDetection}.
The fast detection and verification of possibly threatening events are essential and heavily depend on a good-working epidemic intelligence to handle a large amount of data from various sources.
Especially in the times of the internet, the topicality and quantity of data can be useful to detect even rumors of suspected outbreaks.
As a result, more than 60\% of the initial outbreak reports come from such informal sources \citep{EpiSurv}.
However, filtering this massive amount of data for notorious diseases poses the difficulty to find the right threshold for which events to consider interesting and which to discard.
The required high sensitivity to warn about findings competes with the demand to filter the massive amount of data on the web.

\subparagraph{Indicator-Based Surveillance}
Following a more traditional approach, \textbf{indicator-based-surveillance}\index{indicator-based-surveillance} concerns itself with the interpretation of structured data, using trustworthy, human and non-human related health-based formal sources \citep{EarlyDetection}.
The acquisition of this data is a passive process and follows routines established by the legislator and the public health institute.
These routines follow rules that are disease or syndrome-specific.
Opposed to EBS, indicator-based surveillance is not only responsible for event detection but also for measuring the impact and evaluation of programs \citep{EarlyDetection}.

\subsection{The Robert Koch Institute}
The Robert Koch Institute is the public health institute of Germany.
It consists of several departments which cover responsibilities as microbiological research, monitoring of non-communicable diseases, infectious disease epidemiology, and biological threats intervention.
These departments consist of several units with more specific agendas.

\subsubsection{Signale}
Among those units, the Infectious Disease Data Science Unit carries the main responsibility for providing in-house software solutions for the RKI.
Most units organize themselves as groups based on their different missions.
\textbf{Signale} is one group in the Infectious Disease Data Science Unit, and it is the interface between machine learning and application.
This group provides intuitive access to critical numerical analyzes via dashboards for different use cases within the RKI and is the head of data science.

\subsubsection{EpiLag}
To have an adequate response mechanism that cannot be satisfied by existing surveillance infrastructure, in 2009 the \textbf{epidemiologische L\"ander-Bund-Konferenz} (\gls{EpiLag})  established \citep{Mohr2010}.
Since the federal states of Germany govern health policy, the infrastructure of the health authorities is different for each state.
To still communicate efficiently, the EpiLag established a weekly conference call where the RKI and the federal states are exchanging on events of high importance for the public health which require immediate attention.
The EpiLag gives recommendations on how to handle threats for public health and provides a platform to coordinate undertakings that affect several federal states.

\subsubsection{INIG}
The Information Centre for International Health Protection (\GLS{INIG}) is a young unit that is staffed by epidemiologists from several units to combine expertise from different backgrounds in medicine and public health to provide international epidemiological surveillance for the RKI.
For this, they tasked epidemiologists with reading trusted sources (\ref{INIGsources}) for epidemiological articles to find international events that are particularly important for the public health of Germany.
When INIG members are unsure whether an article describes a noteworthy outbreak, they have the opportunity to consult each other.
This procedure then often leads to a more educated decision whether an outbreak article is interesting.
INIG also supports the work of the EpiLag by providing intelligence about countries that exceed Germany's neighboring states, countries with which the RKI has no official information exchange agreement.
According to former Minister of Health \citeauthor{Grohe2017}, the importance of global health was most evident during the uprise of the Ebola disease and Zika epidemic starting 2015 which made clear that the protection of the German citizen and, most importantly, the aid for affected people require international cooperation.

\section{Machine Learning and Epidemiology}
Epidemiology is a delicate field from the perspective of machine learning.
Assessments and decisions in this field can decide over the survival of people.
Thus, each decision made in epidemiology must in all circumstances avoid overseeing signs for emerging outbreaks.
The necessity for high sensitivity is the strong opposition to non-medical fields in which false decisions would most of the time only reduce the customer's satisfaction.
Hence, the preferred approach in the methodology of epidemiologists is to formulate statistical models explicitly which includes the selection of features/variables and their interaction (top-down).
The general data-driven approach (bottom-up) in classical machine learning is perceived precarious by people from the health sector.
Indeed, while a machine learning model could perform better than a custom formula, the machine learning model's decisions used to be hard to disentangle.
But with the recent improvements in visualization \citep{Arras2017}, and endeavors to tackle biases in machine learning models, those usual reservations can be better approached.
Furthermore, with advancing digitalization of public health, more and better quality data become available \citep{DEMIS}.
Due to that, a bottom-up approach is now more often a valid alternative.

This paradigm shift is particularly apparent in the field of natural language processing.
Up to the first decade of the 21st century, mainly explicitly model translation algorithms were used, called \textbf{rule-based-machine-translation}.
They demanded a behemoth of rules to account not only for two grammatical systems that needed to be aligned but also handle special cases, idioms, and dynamics of language \citep{Bar-Hillel1953, Bar-Hillel1960}.
With the renaissance of deep learning, the bottom-up approached performed much better and required less modeling and at the same time accuracy in translations improved by a manifold for several language combinations \citep{Bengio2003}.
The ease with which already existing labeled data (websites that provide their content in different languages) could be utilized was also exemplary for the advancing digitalization \citep{Macklovitch00}.

\section{Motivation}
The broad idea at the beginning of this project was to utilize natural language
processing (\GLS{NLP}) in a dashboard for epidemiologists who perform epidemiological
surveillance to assist them in their reading intensive work.
The RKI has two groups for epidemiological surveillance that would profit the most from NLP aided tools. The two groups, EpiLag and INIG, analyze numerous textual sources for their work and thus I decided to look into their work to pinpoint whether and how NLP might proof useful for them.

While the EpiLag focuses on disease outbreaks within Germany, INIG is mainly interested in international disease outbreaks. Since the EpiLag is a phone conference and the information provided are from the 16 local health authorities of Germany, it is hard to retrace the origin of all their reported information.
Due to the difficult access to data, they did not pose an ideal candidate for the development of an NLP-driven aid.
INIG, however, is a young project that much more depends on self-acquired intelligence.
Unlike EpiLag, INIG has a clear operating procedure to collect information from well-defined sources.
Each week one person of the INIG team has to read articles from a fixed set of sources and filter out outbreak news that are considered important for German officials like the ministry of health or the military.
The INIG stuff member then fills an Excel sheet with the information from the found article.
The overall process costs around 30 minutes every day which led to the idea to automate this process.

The first goal of the thesis was to automatically put key points of an outbreak article into a database and replace the previous Excel-workflow. Being able to describe the article based on its key points led to the second goal: The usage of articles and their keywords to learn the relevance of an article and then use this knowledge to write a recommendation system to decrease the burden of finding important articles.
Third, when having a functional recommendation system, a further goal was to unravel the epidemiologists' decision process and try to detect biases between the different INIG member.
With a working summarization and relevance scoring pipeline, the goal was to expand the work of INIG to more sources and support the parsing of non-English text.
