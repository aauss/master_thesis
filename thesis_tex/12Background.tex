\chapter{Background}

\section{Working At The RKI}

\section{Natural language Processing}


\subsection{Stop Words}
Assuming we want to analyze text on the level of words, we might be looking for
words that appear more often than others. This information can tell as a lot about
the topic or sentiment of the text. Suppose we find that a text is frequently
mentioning \textit{brexit} more than any other word. Then we can deduct, that this text might
be an news article and that it is talking about one of the most controversial issues
of the European Union. However, we quickly realize that certain words
also appear more frequently than others, without revealing much about the content
of the text. Words like \textit{the, of, than} appear numerously in every English
text independently of the topic or the source. This can make it very different to target
words that are actually important for further analyzes. These words are called \textbf{stop words}. \index{stop words}
Should we want to find, as an example, any adverb in a text, then we would care for (almost) all words.
Every word in this context has some information about the grammatical structure of the text
and thus must be taking into consideration. But, if we should be interested in the topic
or source of the text, then we might end up searching for this semantic information within only a handful of words.
In this case it would only distract our learner, classification algorithm, or other machine
learning tool, if the vast majority of words would not transmit the information
of interest. Therefore, it is common practice to remove stop words.


\subsection{Tokenization}
A token is an abstraction of a piece of information. In \GLS{NLP} this can be
single character, word, punctuation, or sentences. The goal in tokenization is
to split text into meaningful chunks that obey the rules of the language which defined
what a single word or sentence means.
Sentence tokenization already becomes important if a text
includes logical parts (like parapgraphs or chapters) that need to be handled differently.
Word-tokenization is important when you want to apply POS-tagging, followed by NER.
These two methods require words to work correctly.

\subsection{Bag Of Words}
In a machine learning task it is preferable to use the large size of data as a leverage to avoid
doing explicit feature engineering, but to let the algorithm pick the feature.

\subsection{(Disease) Name Entity Recognition}
Name entity recognition (NER) is based in the middle of a NLP pipeline. After
sentences and words have been tokenized, and position-of-speak tagging (POS-
tagging) was applied, it might be important for certain learning algorithms to
infer the entity of the word at hand. Common examples are the distinction of
ambiguous words as apple. In the beginning of a text it could be the fruit or a
billion dollar company.
In the medical field such ambiguities rise not because the proper names are so
indistinguishable form other common words, but because there are many form
how to write a disease name and equally many abbreviations. Thus, disease-NER
is a very important processing step.

\section{Machine Learning}

\subsection{Naive Bayes Classifier}
The naive Bayes classifier (NBC) is a probabilistic classifier. It describes a set
of algorithms capable to learn to infer a label given a set of features. These
algorithms are trained in a supervised fashion.
