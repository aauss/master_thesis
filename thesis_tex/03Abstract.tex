\abstract{%
To ensure permanent responsiveness to emerging disease outbreaks, the Robert Koch Institute (\GLS{RKI}) is the contact and evaluation point for notifications of reportable diseases form health departments and other public health institutes. However, the RKI also needs to search for emerging outbreaks in the vast amount of international disease outbreak news to detect dangerous outbreaks as early as possible. For this, they read the news from several sources every day, summarize the most important articles, and then write reports about them. This time-consuming procedure can be shortened by my developed surveillance tool which summarizes epidemiological news and rates the relevance of articles for the RKI.

Before I could develop the surveillance tool, I needed to acquire data to train machine learning models. The RKI provided a simple Excel sheet that contained all articles considered important in the year 2018 and several columsn mentioning the key points of the articles (e.g. disease, country, confirmed cases \ldots). I preprocessed the Excel sheet and transferred it to a controlled vocabulary. Additionally, I scraped all articles from the most used sources of the epidemiologists and created a labeled data set where articles extracted by the RKI have a textit{relevant} label and all the other articles an textit{irrelevant} label.

To summarize articles, I applied name entity recognition to epidemiological texts, and then trained learning methods with the RKI data to find the most important entities to describe the article- The learner used all sentences of one article as an input to find the most salient sentence-entity combination to extract the key points of the article. The recommendation system was trained based on the whole text in combination with those key entities. Finally, I built a web app to make the surveillance tool accessible so that epidemiologist can shift their focus to writing reports instead of reading and searching articles that are irrelevant to them.
}
