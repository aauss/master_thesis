\chapter{Introduction}
\section{EXAMPLES}
A citation: \cite{example}. There are some handy options for citing publications. It is possible to print just the year of some publication:~\citeyear{example}. It is also possible to print the name of the author(s) in the form ``Author et al.'': \citeauthor{example}. Finally, there is also a command to print the full list of authors of a publication: \citet*{example}.

This is an example glossary reference: \GLS{pi}. \\


\begin{figure}[ht]
    \centering
    \includegraphics{BWUni}
    \caption[Cambridge University BW Logo]{A black and white version of the Cambridge University logo.}
    \label{fig:bwUniLogo}
\end{figure}

\begin{align}
\boldsymbol{x} &= \{x_1, x_2, \dots, x_n\} \\
\mathcal{C} &= \{\mathcal{C}_k \: | \: k \in K \} \\
P(\mathcal{C}_k|\boldsymbol{x}) &= \frac{P(\boldsymbol{x} |\mathcal{C}_k) P(\mathcal{C}_k)} {P(\boldsymbol{x})} \\
P(\mathcal{C}_k|\boldsymbol{x}) &= \frac{P(x_1) |\mathcal{C})
                                   P(x_2 |\mathcal{C}_k) \dots
                                   P(x_n |\mathcal{C}_k)
                                   P(\mathcal{C}_k)}{ P(\boldsymbol{x})} \\
\end{align}

\section{Motivation}
The course goal at the beginning of this project was to utilize natural language
processing (NLP) in a dashboard for the epidemiologists working in the international
surveillance to improve the work flow of their reading intesive work.
The Signale group at the Robert Koch Institute where I wrote my thesis, focused to create custom solutions for different
fields in epidemiology. Two groups at the RKI, the Epilag and the INIG needed
to parse many texts for their work and thus we decided to look into their work
to figure out how NLP could aid them in their everyday work. Both group’s
expertise is epidemiological surveillance. While the Epilag focuses on events
within Germany, INIG is mainly interested in international outbreak news. The
Epilag group exists for a longer time now and has clear working routines that
incorporate several health departments of the Kreise and Länder of Germany.
Even if I would have found a helpful use of NLP in their work-flow I would have
been difficult to consider so many participants of this group that however are
spread across Germany and have different access to resources of the RKI. INIG
however is a young project that much more depends on a variety of text. They
have much less work-power that can be spared to do tedious work opposed to
Epilag where every health department needs to report to the RKI where then
only the bigger picture is put together. Currently one person of the INIG team
is responsible to read a fixed set of articles and filter out outbreak news that are
important. These are then put into a database. This costs around 30 minutes
every day, that should be spent writing assessments on the outbreaks for the
German Ministry of Health. Thus, the idea developed to automate this process
that least required expertise. Therefore, putting key points of a outbreak article
into a database was the first goal of the thesis. Second, being able to describe the
text based on this condensed form lead to the second goal: Use these keywords
to determine the relevance of an article and then use this knowledge to write
and recommendation system.

\subsection{Signale}
The Robert Koch Institute is the public health institute of Germanny- It is
split into several Abteilungen. Most of them are interested in epidemiology.
These Abteilungen are then split into several Fachbereiche. That is necessary
so that every Fachbereich can focus on a specific topic and publish papers in
this area. FG31 where I have been working is the IT FG and the only one
that is working on Software solutions for the RKI. We stick out, since we are
the interface between statistics and application. We create dashboards and are
relatively independent from the other research and work done.

\subsection{INIG}
Also within the same Abteilung there are many groups that combine their
expertise of certain diseases classes to form a super groups that does international
surveillance. Their advantage is that they have expertise knowledge in different
parts of infectious disease medicine- When they are reading news articles and they
are unsure whether and article is interesting or not, they have the opportunity
to consult each other which then leads to a very educated decision whether an
outbreak article is interesting or not.

