\chapter{Introduction}
% \section{EXAMPLES}
% A citation: \cite{example}. There are some handy options for citing publications. It is possible to print just the year of some publication:~\citeyear{example}. It is also possible to print the name of the author(s) in the form ``Author et al.'': \citeauthor{example}. Finally, there is also a command to print the full list of authors of a publication: \citet*{example}.
%
% This is an example glossary reference: \GLS{pi}. \\
%
%
% \begin{figure}[ht]
%     \centering
%     \includegraphics{BWUni}
%     \caption[Cambridge University BW Logo]{A black and white version of the Cambridge University logo.}
%     \label{fig:bwUniLogo}
% \end{figure}


\section{Epidemiological Surveillance at the Robert Koch Institute}

\subsection{Epidemiology}
According to the World Health Organization (\GLS{WHO}) \citep{WHOepi} epidemiology is a large field of studies that focus
on the distribution and the determinants of health-related signals which include the (subsequent)
containment of a disease spread. This means, that epidemiology is focusing on infectious diseases which can be contagious (direct transmission) or communicable (via a vector like insects). However, infectious diseases are ubiquitous and are most of the time sufficiently dealt with by a functional public health system. To therefore be more specific, epidemiology focuses on diseases occurring as part of epidemics, pose a threat for epidemics, or significantly burden the public health system. Epidemics can be expressed as the occurrences of cases in excess of what is normally expected given time and region. To classify and investigate epidemics, epidemiology is not limiting itself to certain methods but include several families of methods like surveillance and descriptive studies.


\subsubsection{Epidemiological Surveillance}

Though epidemiology is a broad field, surveillance is probably its most indispensable tool. Through surveillance,
epidemiologists are able to detect all hazard (nuclear, chemical, biological \ldots) threats for public health. In the course of this thesis, I will focus only on threats by infectious diseases. The detection of threats is important for the rapid investigation and response of lethal disease spreads. But, also the monitoring of an outbreak is important to allocate resources of the public health systems effectively \citep{EarlyDetection}.

Hints for an outbreak can be given by an increased amount of reported cases of a dangerous infection. But also changed circumstances are sometimes inevitably linked to outbreaks e.g. a loss of proper sanitation is often linked to cholera. Besides the traditional disease-based surveillance that processes laboratory confirmations, syndromic surveillance, and external sources external factors like weather, attendance monitoring at school and workplace, social media, and the web\citep{EarlyDetection} can be equally informative. Epidemiological surveillance is especially powerful in the times of Internet, where even rumors of suspected outbreaks can be found in the vast amount of data to then be analyzed for their importance \citep{EpiSurv}. However, filtering the web for notorious diseases poses the difficulty to find the right threshold for which data to consider interesting and which to discard. The required high sensitivity to warn about findings competing with the demand to filter the huge amount of data on the web. Thus, it requires higher education and special training for an epidemiologist to find and apply the threshold to news articles (of official and unofficial sources) to find the next fatal outbreak ahead of time.



\paragraph{Event and Indicator Based Surveillance}

\subparagraph{Epidemic Intelligence}
Independent of the form of surveillance, epidemiologists need a data management system and a platform for information sharing. This is called \textbf{epidemic intelligence}\index{epidemic intelligence} and facilitates the systematic collection, analysis, and detection of events\citep{EarlyDetection}. As such, it is an important building block for the event and indicator-based surveillance.

\subparagraph{Event-Based Surveillance}
The monitoring of information not generated by the public health system and its analysis is called \textbf{event-based-surveillance}\index{event-based-surveillance}(\gls{EBS}) and is the speed determining factor in epidemiological surveillance. Only with EBS, epidemiologists are able to detect and report events before the recognition of human cases in the reporting system of the public health system\citep{EarlyDetection}. The fast detection and verification of possibly threatening events are important and heavily depends on a good-working epidemic intelligence to handle a large amount of data from various sources.

\subparagraph{Indicator-Based Surveillance}
Following a more traditional approach, \textbf{indicator-based-surveillance}\index{indicator-based-surveillance} concerns itself with the interpretation of structured data, using trustworthy, human and non-human related health-based formal sources\citep{EarlyDetection}. The acquisition of this data is a passive process and follows routines established by the legislator and/or the public health institute. These routines follow rules that are disease or syndrome-specific. Opposed to EBS, indicator-based surveillance is not only responsible for event detection but for the impact of evaluation of programs \citep{EarlyDetection}.


\subsection{The Robert Koch Institute and Signale}
The Robert Koch Institute is the public health institute of Germany - It is
split into several departments covering responsibilities ranging from infectious diseases, infectious epidemiology, methodology research, and biological threats intervention.
These departments are then split into several faculties. The infectious epidemiology department has its own IT faculty,
where I have been working for my thesis. It carries the main responsibility for providing in-house software solutions for the RKI. This faculty is then again split into several groups that are specialized in different topics.
Our group, Signale, is the interface between machine learning and application.
We provided intuitive access to important numerical analyzes via dashboards for different use cases within the RKI.

\subsubsection{EpiLag}
To have an adequate response mechanism that cannot be satisfied by existing surveillance infrastructure, in 2009 The epidemiologische L\"ander-Bund-Konferenz (\gls{EpiLag}) has been found \citep{Mohr2010}. Since health politics is governed by the federal states of Germany, the infrastructure of the health authorities is different for each state. To still communicate efficiently, the EpiLag established a weekly conference call where the RKI and the federal states are exchanging events of high importance for the public health which require immediate attention. The EpiLag gives recommendations on how to handle threats for public health and provides a platform to coordinate scientific or political undertakings that affect several federal states.

\subsubsection{INIG}
The Information Centre for International Health Protection (\GLS{INIG}) is a young project staffed by epidemiologists across several faculties to combine expertise from different backgrounds in medicine and public health to provide international epidemiological surveillance for the RKI. For this, they have designated epidemiologists reading trusted sources for epidemiological articles to find events that are particularly important for the public health of Germans. When INIG members are unsure whether an article describes a noteworthy outbreak, they have the opportunity to consult each other. This then often leads to a more educated decision whether an outbreak article is interesting. INIG hereby complements the work of the EpiLag which mostly does not take countries into account that exceed Germany's neighboring states. The importance of global health was most prominent during the uprise of the Ebola disease epidemic starting 2015 and made clear that protection of the German citizen and the aid for affected people only works with international cooperation \citep{Grohe2017}.

\section{Machine Learning in Epidemiolgy}
Epidemiology is a delicate field, where assessments of situations can decide over the survival of people. Unrecognized or untamable disease outbreaks are to be avoided in any case. This is the strong opposition to non-medical fields in which false decisions would probably only reduce customer's satisfaction. Thus, the preferred approach in the statistical methodology of epidemiologists is to explicitly formulate those assessments. This mainly includes the explicit formulation of features/variables and their interaction (top-down). The general data-driven approach (bottom-up) in classical machine learning is perceived precarious. Indeed, while a machine learning model could perform better than a custom formula, the machine learning model's decision are hard to disentangle. But with the recent improvements in visualization, and handling biases in machine learning models, those usual reservations can be better approached on the one hand. On the other hand, with advancing digitalization (DEMIS) of the public health, more and better quality data become available \citep{DEMIS}. The trade-off between the top-down approach and the detection of highly complex relations in the data as in the bottom-up approach shifts to the former. This trade-off becomes particularly clear in the field of natural language processing. Up to the first decade of the 21st century, mainly explicitly model translation algorithms were used, called rule-based-machine-translation. They demanded a behemoth of rules to account not only for two grammatical systems that needed to be aligned but also handle special cases, idioms, and dynamics of language \citep{Bar-Hillel1953, Bar-Hillel1960}. With the renaissance of deep learning, the bottom-up approached performed much better and required less modeling and accuracy in translations improved by a manifold for several language combinations languages \citep{Bengio2003}. The ease with which already existing labeled data (webpages that provide their content in different languages) could be utilized was also exemplary for the advancing digitalization \citep{Macklovitch00}.



\section{Motivation}
The course goal at the beginning of this project was to utilize natural language
processing (\GLS{NLP}) in a dashboard for epidemiologists who perform epidemiological
surveillance to improve their reading intensive work.
The Signale group at the RKI has not yet integrated NLP for their dashboards, although there would be a strong benefit not only for the epidemiological surveillance but also for other groups. The RKI has two groups for epidemiological surveillance that would profit the most from NLP aided tools. The two groups, EpiLag and INIG, analyze numerous textual sources for their work and thus I decided to look into their work to pinpoint how NLP could aid them in their work.

While the EpiLag focuses on disease outbreaks within Germany, INIG is mainly interested in international disease outbreaks. The
EpiLag has been existing for several years and already established a workflow that is heavily linked to authorities with other infrastructures. For this reason, they did not pose an ideal candidate for the development of an NLP-driven aid. Besides the lack of a promising improvement of their work through NLP, the strong differences in the infrastructure of the participants of the EpiLag made it unfavorable to find a solution that could help all them.
INIG, however, is a young project that much more depends on self-acquired sources. Unlike EpiLag, INIG
has no support from other institutions to collect information from these sources.
Each week one person of the INIG team has to read articles from a fixed set of sources and filter out outbreak news that is
important or considered interesting for German officials like the ministry of health or the military. These are then put into an Excel sheet. This process costs around 30 minutes
every day, that could be spent writing needed assessments on the found outbreaks. Thus, the idea developed to automate this process.

Putting key points of an outbreak article
into a database was the first goal of the thesis. Being able to describe the article based on its key points lead to the second goal: Usage of these keywords to learn the relevance of an article and then use this knowledge to write
and recommendation system to decrease the burden of reading unimportant articles.
