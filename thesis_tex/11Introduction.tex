\chapter{Introduction}
\section{EXAMPLES}
A citation: \cite{example}. There are some handy options for citing publications. It is possible to print just the year of some publication:~\citeyear{example}. It is also possible to print the name of the author(s) in the form ``Author et al.'': \citeauthor{example}. Finally, there is also a command to print the full list of authors of a publication: \citet*{example}.

This is an example glossary reference: \GLS{pi}. \\


\begin{figure}[ht]
    \centering
    \includegraphics{BWUni}
    \caption[Cambridge University BW Logo]{A black and white version of the Cambridge University logo.}
    \label{fig:bwUniLogo}
\end{figure}

\begin{align}
\boldsymbol{x} &= \{x_1, x_2, \dots, x_n\} \\
\mathcal{C} &= \{\mathcal{C}_k \: | \: k \in K \} \\
P(\mathcal{C}_k|\boldsymbol{x}) &= \frac{P(\boldsymbol{x} |\mathcal{C}_k) P(\mathcal{C}_k)} {P(\boldsymbol{x})} \\
P(\mathcal{C}_k|\boldsymbol{x}) &= \frac{P(x_1) |\mathcal{C})
                                   P(x_2 |\mathcal{C}_k) \dots
                                   P(x_n |\mathcal{C}_k)
                                   P(\mathcal{C}_k)}{ P(\boldsymbol{x})} \\
\end{align}

\section{Motivation}
The course goal at the beginning of this project was to utilize natural language
processing (\GLS{NLP}) in a dashboard for epidemiologists who work in the international
surveillance to improve their work flow in their reading intesive work.
The Signale group at the Robert Koch Institute (\GLS{RKI}) where I wrote my thesis at, focuses to create custom solutions for different
fields in epidemiology. Two groups at the \GLS{RKI}, the epidemiologische L\"ander-Bund-Konferenz (\GLS{EpiLag})
and the Information Centre for International Health Protection (\GLS{INIG}),
analyse several textual sources for their work and thus we decided to look into their work
to figure out how \GLS{NLP} could aid them in their work. Both group’s
expertise is epidemiological surveillance. While the EpiLag focuses on events
within Germany, INIG is mainly interested in international disease outbreak. The
EpiLag group has been existing for a longer time than INIG and has clear working routines that
incorporate several health departments of the counties and federal states of Germany.
Even if evaluations of their working routings would have revealed suitable applications of \GLS{NLP}, it would have
been difficult to consider so many participants of this group that are
spread across Germany and have different access to resources of the \GLS{RKI} into a meaningful dashboards.
INIG, however, is a young project that much more depends on self acquired sources. Unlike EpiLag, INIG
has no support from other institutions, but depends on self aquired sources.
Each week one person of the INIG team has to read articles from a fixed set of sources and filter out outbreak news that are
important or considered interesting for German officials. These are then put into an Excel sheet. This process costs around 30 minutes
every day, that could be spent writing assessments on the outbreaks for the
German Ministry of Health. Thus, the idea developed to automate this process. Putting key points of a outbreak article
into a database was the first goal of the thesis. Second, being able to describe the
text based on this condensed form lead to the second goal: Use these keywords
to determine the relevance of an article and then use this knowledge to write
and recommendation system.

\subsection{Signale}
The \GLS{RKI} is the public health institute of Germanny - It is
split into several departments. Most of them are interested in epidemiology.
These departments are then split into several faculties. The epidemiological IT faculty,
where I have been working is, is the only one that is working on software solutions for the \GLS{RKI}.
This faculty is then again split several groups that specialzed in different topics.
Our group, Signale, is the interface between statistics and application.
We create dashboards and are relatively independent from the other research and work of the IT faculty.

\subsection{INIG}
Also within the same departments there are many groups that combine their
expertise of certain diseases classes to form a super groups that does international
surveillance. Their advantage is that they have expertise knowledge in different
parts of infectious disease medicine. When they are reading news articles and they
are unsure whether and article is interesting or not, they have the opportunity
to consult each other which then leads to a very educated decision whether an
outbreak article is interesting or not.

\subsection{Epidemiology}
