\chapter{Conclusion}
The main objective was to show, that it is possible to utilize already an available resources, the incident database, apparent at the RKI using novel NLP methodology to improve epidemiological surveillance. Thereby, the outcome of this thesis also proves that EBS even with smaller resources possible (opposite to EIOS). Furthermore, Finally, I was able to show that my developed tool for summarization and relevance scoring can be made available in a scalable way which is of high interest, since the RKI also shares expertise with other countries.

Although the amount and quality of the data was below-average, and the first usability test of the corpus did not look promising (Fig \ref{fig:t-sne}), the classifier performed surprisingly well. Furthermore, I was positively surprised that even after going through several sources and possible directions of the topic, the completion of a self-contained application was possible.

Of course much more work is necessary to bring the Aussinator into production. The relevance score needs to be explainable, and the key word extraction of the date and the count is not satisfactorily.

However, these points can be tackled and also those that were clear from beginning on such as the availability in several languages and the avoidance and the uncovering of biases.


Due to BERT and ELMO, Facebook cross language, in the future it will be much easier to implement cross language systems.

While EpiTator does not provide such solution yet, it would be necessary to have large data amounts for a good classification to then put this into a multi language system.

Also tackling biases and personal preferences is important to continue this project and make it accessible to more people.

To win the approval of epidemiologists that only conduct IBS, it will be important to show that they can trust the algorithm. Therefore, it would be interesting to apply the CNN visualization when demanded to increase the trustworthiness of the algorithm.

Finally, since the app is hosted in the cloud, the next step would be to give several people access, to this function, have a base network learn the classification and then use transfer learning, to adopt the network to indivual preferences.

Try more classifier.
