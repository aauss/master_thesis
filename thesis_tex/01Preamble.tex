%%%%%%%%%%%%%%%%%%%%%%%%%%%%%%%%%%%%%%%%%%%%%%%%%%%%%%%%%%%%%%%%%%%%%%%%%%%%%%%%
%% Some custom packages
%%
\RequirePackage{amsmath}
\RequirePackage{amssymb}
\RequirePackage{xspace}

\DeclareMathOperator*{\argmax}{arg\,max}



%%%%%%%%%%%%%%%%%%%%%%%%%%%%%%%%%%%%%%%%%%%%%%%%%%%%%%%%%%%%%%%%%%%%%%%%%%%%%%%%
%% Fonts (like different typewriter fonts etc.)
%%
%\RequirePackage[scaled=.87]{couriers}
%\RequirePackage[T1]{fontenc}
%\renewcommand\rmdefault{psb}




%%%%%%%%%%%%%%%%%%%%%%%%%%%%%%%%%%%%%%%%%%%%%%%%%%%%%%%%%%%%%%%%%%%%%%%%%%%%%%%%
%% Style (Changing the visual style of chapter headings and stuff.)
%%
\RequirePackage{titlesec}
% [Fixes issue #34 (see https://github.com/cambridge/thesis/issues/34). Solution from: http://tex.stackexchange.com/questions/299969/titlesec-loss-of-section-numbering-with-the-new-update-2016-03-15
\RequirePackage{etoolbox}
\makeatletter
\patchcmd{\ttlh@hang}{\parindent\z@}{\parindent\z@\leavevmode}{}{}
\patchcmd{\ttlh@hang}{\noindent}{}{}{}
\makeatother
% end of issue #34 fix]
\newcommand{\PreContentTitleFormat}{\titleformat{\chapter}[display]{\scshape\Large}
{\Large\filleft\MakeUppercase{\chaptertitlename} \Huge\thechapter}
{1ex}
{}
[\vspace{1ex}\titlerule]}
\newcommand{\ContentTitleFormat}{\titleformat{\chapter}[display]{\scshape\huge}
{\Large\filleft\MakeUppercase{\chaptertitlename} \Huge\thechapter}
{1ex}
{\titlerule\vspace{1ex}\filright}
[\vspace{1ex}\titlerule]}
\newcommand{\PostContentTitleFormat}{\PreContentTitleFormat}
\PreContentTitleFormat




%%%%%%%%%%%%%%%%%%%%%%%%%%%%%%%%%%%%%%%%%%%%%%%%%%%%%%%%%%%%%%%%%%%%%%%%%%%%%%%%
%% Bibliography (special style etc.)
%%
% \RequirePackage[numbers,sort&compress]{natbib}
\RequirePackage[square]{natbib}




%%%%%%%%%%%%%%%%%%%%%%%%%%%%%%%%%%%%%%%%%%%%%%%%%%%%%%%%%%%%%%%%%%%%%%%%%%%%%%%%
%% Theorems, definitions, and examples
%%
\RequirePackage{amsthm}
\theoremstyle{definition}
\newtheorem{definition}{Definition}[chapter]
%% Support for `Examples` (provides a counter for examples, the possibility to
%% label and reference them etc.)
%%
\newtheorem{example}{Example}[chapter]




%%%%%%%%%%%%%%%%%%%%%%%%%%%%%%%%%%%%%%%%%%%%%%%%%%%%%%%%%%%%%%%%%%%%%%%%%%%%%%%%
%% Captions: This makes captions of figures use a boldfaced small font.
%%
\RequirePackage[small,bf]{caption}




%%%%%%%%%%%%%%%%%%%%%%%%%%%%%%%%%%%%%%%%%%%%%%%%%%%%%%%%%%%%%%%%%%%%%%%%%%%%%%%%
%% Subfigure (note: this must be included after the `caption` package).
%%
\RequirePackage{subfig}





%%%%%%%%%%%%%%%%%%%%%%%%%%%%%%%%%%%%%%%%%%%%%%%%%%%%%%%%%%%%%%%%%%%%%%%%%%%%%%%%
%% Graphics (we set the central folder for all included graphics to
%% `./Figures/`)
%%
\graphicspath{{./Figures/}}




%%%%%%%%%%%%%%%%%%%%%%%%%%%%%%%%%%%%%%%%%%%%%%%%%%%%%%%%%%%%%%%%%%%%%%%%%%%%%%%%
%% Content commands
%%




%%%%%%%%%%%%%%%%%%%%%%%%%%%%%%%%%%%%%%%%%%%%%%%%%%%%%%%%%%%%%%%%%%%%%%%%%%%%%%%%
%% Glossary entries
%%
\newglossaryentry{pi}{
    name={\ensuremath{\pi}},
    sort={pi},
    description={ratio of the circumference of a circle to the diameter}
}

\newglossaryentry{NLP}
{
  name={NLP},
  description={natural language processing}
}

\newglossaryentry{RKI}
{
  name={RKI},
  description={Robert Koch Institute}
}

\newglossaryentry{EpiLag}
{
  name={EpiLag},
  text={EpiLag},
  description={Epidemiologische Bund-L\"ander-Lagekonferenz \\
               \textit{- epidemiological federal state-state conference}}
}

\newglossaryentry{INIG}
{
  name={INIG},
  description={The Information Centre for International Health Protection}
}

\newglossaryentry{WHO}
{
  name={WHO},
  description={World Health Organization}
}

\newglossaryentry{NER}
{
  name={NER},
  description={name entity recognition}
}

\newglossaryentry{POS-tagging}
{
  name={POS-tagging},
  text={POS-tagging},
  description={position of speech tagging}
}

\newglossaryentry{NBC}
{
  name={NBC},
  description={naive Bayes classifier}
}

\newglossaryentry{regex}
{
  name={regex},
  description={regular expression}
}

\newglossaryentry{EBS}
{
  name={EBS},
  description={event-based surveillance}
}

\newglossaryentry{IR}
{
  name={IR},
  description={information retrieval}
}

\newglossaryentry{ML}
{
  name={ML},
  description={information retrieval}
}
