%%%%%%%%%%%%%%%%%%%%%%%%%%%%%%%%%%%%%%%%%%%%%%%%%%%%%%%%%%%%%%%%%%%%%%%%%%%%%%%%
%% Some custom packages
%%
\RequirePackage{amsmath}
\RequirePackage{amssymb}
\RequirePackage{xspace}
\RequirePackage{tikz}
\RequirePackage{booktabs}
\RequirePackage{textcomp}
\RequirePackage{fancyvrb}
\usepackage{minted}
\usepackage{algorithm2e}
%friendly, perldoc, tango
\usemintedstyle{tango}

\LetLtxMacro{\cminted}{\minted}
\let\endcminted\endminted
\RecustomVerbatimEnvironment{Verbatim}{BVerbatim}{}


\DeclareMathOperator*{\argmax}{arg\,max}
\DeclareMathOperator*{\argmin}{arg\,min}
\def\layersep{2.5cm} % Defines distance for ANN figure
\newcommand{\ra}[1]{\renewcommand{\arraystretch}{#1}}



%%%%%%%%%%%%%%%%%%%%%%%%%%%%%%%%%%%%%%%%%%%%%%%%%%%%%%%%%%%%%%%%%%%%%%%%%%%%%%%%
%% Fonts (like different typewriter fonts etc.)
%%
%\RequirePackage[scaled=.87]{couriers}
%\RequirePackage[T1]{fontenc}
%\renewcommand\rmdefault{psb}




%%%%%%%%%%%%%%%%%%%%%%%%%%%%%%%%%%%%%%%%%%%%%%%%%%%%%%%%%%%%%%%%%%%%%%%%%%%%%%%%
%% Style (Changing the visual style of chapter headings and stuff.)
%%
\RequirePackage{titlesec}
% [Fixes issue #34 (see https://github.com/cambridge/thesis/issues/34). Solution from: http://tex.stackexchange.com/questions/299969/titlesec-loss-of-section-numbering-with-the-new-update-2016-03-15
\RequirePackage{etoolbox}
\makeatletter
\patchcmd{\ttlh@hang}{\parindent\z@}{\parindent\z@\leavevmode}{}{}
\patchcmd{\ttlh@hang}{\noindent}{}{}{}
\makeatother
% end of issue #34 fix]
\newcommand{\PreContentTitleFormat}{\titleformat{\chapter}[display]{\scshape\Large}
{\Large\filleft\MakeUppercase{\chaptertitlename} \Huge\thechapter}
{1ex}
{}
[\vspace{1ex}\titlerule]}
\newcommand{\ContentTitleFormat}{\titleformat{\chapter}[display]{\scshape\huge}
{\Large\filleft\MakeUppercase{\chaptertitlename} \Huge\thechapter}
{1ex}
{\titlerule\vspace{1ex}\filright}
[\vspace{1ex}\titlerule]}
\newcommand{\PostContentTitleFormat}{\PreContentTitleFormat}
\PreContentTitleFormat




%%%%%%%%%%%%%%%%%%%%%%%%%%%%%%%%%%%%%%%%%%%%%%%%%%%%%%%%%%%%%%%%%%%%%%%%%%%%%%%%
%% Bibliography (special style etc.)
%%
% \RequirePackage[numbers,sort&compress]{natbib}
\RequirePackage[square]{natbib}




%%%%%%%%%%%%%%%%%%%%%%%%%%%%%%%%%%%%%%%%%%%%%%%%%%%%%%%%%%%%%%%%%%%%%%%%%%%%%%%%
%% Theorems, definitions, and examples
%%
\RequirePackage{amsthm}
\theoremstyle{definition}
\newtheorem{definition}{Definition}[chapter]
%% Support for `Examples` (provides a counter for examples, the possibility to
%% label and reference them etc.)
%%
\newtheorem{example}{Example}[chapter]




%%%%%%%%%%%%%%%%%%%%%%%%%%%%%%%%%%%%%%%%%%%%%%%%%%%%%%%%%%%%%%%%%%%%%%%%%%%%%%%%
%% Captions: This makes captions of figures use a boldfaced small font.
%%
\RequirePackage[small,bf]{caption}




%%%%%%%%%%%%%%%%%%%%%%%%%%%%%%%%%%%%%%%%%%%%%%%%%%%%%%%%%%%%%%%%%%%%%%%%%%%%%%%%
%% Subfigure (note: this must be included after the `caption` package).
%%
\RequirePackage{subfig}





%%%%%%%%%%%%%%%%%%%%%%%%%%%%%%%%%%%%%%%%%%%%%%%%%%%%%%%%%%%%%%%%%%%%%%%%%%%%%%%%
%% Graphics (we set the central folder for all included graphics to
%% `./Figures/`)
%%
\graphicspath{{./Figures/}}




%%%%%%%%%%%%%%%%%%%%%%%%%%%%%%%%%%%%%%%%%%%%%%%%%%%%%%%%%%%%%%%%%%%%%%%%%%%%%%%%
%% Content commands
%%




%%%%%%%%%%%%%%%%%%%%%%%%%%%%%%%%%%%%%%%%%%%%%%%%%%%%%%%%%%%%%%%%%%%%%%%%%%%%%%%%
%% Glossary entries
%%

\newglossaryentry{NLP}
{
  name={NLP},
  description={natural language processing}
}

\newglossaryentry{RKI}
{
  name={RKI},
  description={Robert Koch Institute}
}

\newglossaryentry{EpiLag}
{
  name={EpiLag},
  text={EpiLag},
  description={Epidemiologische Bund-L\"ander-Lagekonferenz \\
               \textit{- Epidemiological Federal State-State Conference}}
}

\newglossaryentry{INIG}
{
  name={INIG},
  description={Informationsstelle für Internationalen Gesundheitsschutz\\
  \textit{- The Information Centre for International Health Protection}}
}

\newglossaryentry{WHO}
{
  name={WHO},
  description={World Health Organization}
}

\newglossaryentry{NER}
{
  name={NER},
  description={named-entity recognition}
}

\newglossaryentry{POS-tagging}
{
  name={POS-tagging},
  text={POS-tagging},
  description={position of speech tagging}
}

\newglossaryentry{NBC}
{
  name={NBC},
  description={naive Bayes classifier}
}

\newglossaryentry{regex}
{
  name={regex},
  description={regular expression}
}

\newglossaryentry{EBS}
{
  name={EBS},
  description={event-based surveillance}
}

\newglossaryentry{IR}
{
  name={IR},
  description={information retrieval}
}

\newglossaryentry{ML}
{
  name={ML},
  description={machine learning}
}

\newglossaryentry{MLP}
{
  name={MLP},
  description={multilayer perceptron}
}

\newglossaryentry{CNN}
{
  name={CNN},
  description={convolutional neural network}
}

\newglossaryentry{HTML}
{
  name={HTML},
  description={Hypertext Markup Language}
}

\newglossaryentry{CSS}
{
  name={CSS},
  description={Cascading Style Sheets}
}

\newglossaryentry{REST}
{
  name={REST},
  description={Representational State Transfer}
}

\newglossaryentry{tf-idf}
{
  name={tf-idf},
  description={term frequency-inverse document frequency}
}

\newglossaryentry{NLTK}
{
  name={NLTK},
  description={Natural Language Tool Kit}
}

\newglossaryentry{EDB}
{
  name={EDB},
  description={Ereignisdatenbank\\
               \textit{- Incident Database}}
}

\newglossaryentry{RASFF}
{
  name={RASFF},
  description={Rapid Alert System for Food and Feed}
}

\newglossaryentry{EPIS}
{
  name={EPIS},
  description={Epidemic Intelligence Information System}
}

\newglossaryentry{EU}
{
  name={EU},
  description={European Union}
}

\newglossaryentry{IfSG}
{
  name={IfSG},
  description={Infektionsschutzgesetz\\
               \textit{- Protection Against Infection Act}}
}

\newglossaryentry{WHO DON}
{
  name={WHO DON},
  description={World Health Organization Disease Outbreak News}
}

\newglossaryentry{IBA}
{
  name={IBA},
  description={Index of Balanced Accuracy}
}

\newglossaryentry{ADASYN}
{
  name={ADASYN},
  description={Adaptive Synthetic Sampling Approach for Imbalanced Learning}
}

\newglossaryentry{GRITS}
{
  name={GRITS},
  description={Global Rapid Identification Tool System}
}

\newglossaryentry{SVM}
{
  name={SVM},
  description={support vector machine}
}
